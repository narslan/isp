\documentclass{article}

%encoding
%--------------------------------------
\usepackage[utf8]{inputenc}
\usepackage[T1]{fontenc}
%----------------------------

%fonts
%--------------------------------------
%---------------
% \setromanfont{Times New Roman}
% \setsansfont{Arial}
% \setmonofont[Color={0019D4}]{Courier New}
%
\usepackage{libertine}
\usepackage{beramono}

%units
%--------------------------------------
\usepackage{siunitx}

%drawing
%------------------------------------
\usepackage{tikz} % To generate the plot from csv
\usepackage{pgfplots}
\usepackage{pgfplotstable}
\pgfplotsset{compat=1.11}
\usepgfplotslibrary{units}


\begin{document}


%\pgfplotstableread{spec.dat}\loadedtable
\pgfplotsset{
plotl/.style={blue,no marks,ultra thick,domain=480:780,samples=50},
plotd/.style={red,no marks,ultra thick,domain=-2:2,samples=50}
}
\begin{figure}[h!]
  \begin{center}

    \begin{tikzpicture}
      \begin{axis}[
          width=\linewidth, % Scale the plot to \linewidth
          xlabel= $E$, % Set the labels
          ylabel= $i$,
          x unit= \si{\milli\volt} , % Set the respective units
          y unit= \si{\milli\ampere},
          no marks,
         legend style={at={(0.5,-0.2)},anchor=north},
          ]

    \addplot
         table[x=nm,y=int,col sep=comma] {spec.dat};
%\addplot[plotl]{0.15*exp(-x/0.5)/(0.5*sinh(4)}; %\addlegendentry{$T = 0.5$}\label{plot:probL}

        %\addplot
        % add a plot from table; you select the columns by using the actual name in
        % % the .csv file (on top)
        % table[x=Potential,y=Current,col sep=comma] {table4.csv};
        % \addplot
        % table[x=Potential,y=Current,col sep=comma] {table5.csv};
        % \addplot
        % table[x=Potential,y=Current,col sep=comma] {table6.csv};
        % \addplot
        % table[x=Potential,y=Current,col sep=comma] {table7.csv};
        % \addplot
        % table[x=Potential,y=Current,col sep=comma] {table8.csv};
        % \addplot
        % table[x=Potential,y=Current,col sep=comma] {table9.csv};
        % \addplot
        % table[x=Potential,y=Current,col sep=comma] {table10.csv};
        % \legend{0.005 , 0.01, 0.025, 0.05, 0.1, 0.25, 0.5}
      \end{axis}
    \end{tikzpicture}
    \caption{Aufgabe 3 Überlappung}
  \end{center}
\end{figure}


\end{document}