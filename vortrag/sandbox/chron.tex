\documentclass{article}
\usepackage[german]{babel}

%\usepackage[activate={true,nocompatibility},final,tracking=true,kerning=true,spacing=true,factor=1100,stretch=10,shrink=10]{microtype}
%\usepackage{paratype}
\usepackage{libertine}
%\usepackage[scaled]{berasans}
\renewcommand*\familydefault{\sfdefault}  %%
\usepackage[utf8]{inputenc}

\usepackage[T1]{fontenc}

\usepackage{microtype}
\usepackage{xcolor}
\usepackage{pgffor}
\usepackage[version=4]{mhchem}
%\usetheme{beamertheme-bjeldbak/beamerthemebjeldbak}
\usepackage{tikz}
\usepackage{tikzorbital}
\usetikzlibrary{shapes.callouts} 

\pgfkeys{%
    /calloutquote/.cd,
    width/.code                   =  {\def\calloutquotewidth{#1}},
    position/.code                =  {\def\calloutquotepos{#1}}, 
    author/.code                  =  {\def\calloutquoteauthor{#1}},
    /calloutquote/.unknown/.code   =  {\let\searchname=\pgfkeyscurrentname
                                 \pgfkeysalso{\searchname/.try=#1,                                
    /tikz/\searchname/.retry=#1},\pgfkeysalso{\searchname/.try=#1,
                                  /pgf/\searchname/.retry=#1}}
                            }  


\newcommand\calloutquote[2][]{%
       \pgfkeys{/calloutquote/.cd,
         width               = 5cm,
         position            = {(0,-1)},
         author              = {}}
  \pgfqkeys{/calloutquote}{#1}                   
  \node [rectangle callout,callout relative pointer={\calloutquotepos},text width=\calloutquotewidth,/calloutquote/.cd,
     #1] (tmpcall) at (0,0) {#2};
  \node at (tmpcall.pointer){\calloutquoteauthor};    
}  


\begin{document}

\begin{tikzpicture}
%\draw [->,ultra thick] (-1,-2) --   (-1,4) node[above] {\Large Energie};
%
\drawLevel[pos = {(0,0)},    width = 1]{laf1};
\drawLevel[pos = {(1.2,0)},  width = 1]{laf2};
\drawLevel[pos = {(2.4,0)},  width = 1]{laf3};
\drawLevel[pos = {(3.6,0)},  width = 1]{laf4};
\drawLevel[pos = {(4.8,0)},  width = 1]{laf5};
\drawLevel[pos = {(6,0)},    width = 1]{laf6};
\drawLevel[pos = {(7.2,0)},  width = 1]{laf7};
\drawLevel[pos = {(8.4,0)},  width = 1]{laf8};
%\calloutquote[author=Russelsunders,width=0.5*\linewidth,position={(0,-1)},fill=green!30,rounded corners]{$\ce{^{2S+1}_{}L_J}$};

\drawLevel[elec = up, pos = {(0,1)},    width = 1]{cef1};
\drawLevel[pos = {(1.2,1)},  width = 1]{cef2};
\drawLevel[pos = {(2.4,1)},  width = 1]{cef3};
\drawLevel[pos = {(3.6,1)},  width = 1]{cef4};
\drawLevel[pos = {(4.8,1)},  width = 1]{cef5};
\drawLevel[pos = {(6,1)},  width = 1]{cef6};
\drawLevel[pos = {(7.2,1)},  width = 1]{cef7};
\drawLevel[pos = {(8.4,1)},  width = 1]{cef8};


\drawLevel[elec = up, pos = {(0,2)},    width = 1]{prf1};
\drawLevel[elec = up, pos = {(1.2,2)},  width = 1]{prf2};
\drawLevel[pos = {(2.4,2)},  width = 1]{prf3};
\drawLevel[pos = {(3.6,2)},  width = 1]{prf4};
\drawLevel[pos = {(4.8,2)},  width = 1]{prf5};
\drawLevel[pos = {(6,2)},  width = 1]{prf6};
\drawLevel[pos = {(7.2,2)},  width = 1]{prf7};
\drawLevel[pos = {(8.4,2)},  width = 1]{prf8};


\drawLevel[elec = up, pos = {(0,3)},    width = 1]{ndf1};
\drawLevel[elec = up, pos = {(1.2,3)},  width = 1]{ndf2};
\drawLevel[elec = up, pos = {(2.4,3)},  width = 1]{ndf3};
\drawLevel[pos = {(3.6,3)},  width = 1]{ndf4};
\drawLevel[pos = {(4.8,3)},  width = 1]{ndf5};
\drawLevel[pos = {(6,3)},  width = 1]{ndf6};
\drawLevel[pos = {(7.2,3)},  width = 1]{ndf7};
\drawLevel[pos = {(8.4,3)},  width = 1]{ndf8};




%\drawLevel[pos = {(0,2)},    width = 1]{5d0};
%\drawLevel[pos = {(1.2,2)},  width = 1]{5d1};
%\drawLevel[pos = {(2.4,2)},  width = 1]{5d2};
%\drawLevel[pos = {(3.6,2)},  width = 1]{5d3};
%\drawLevel[pos = {(4.8,2)},  width = 1]{5d4};
\node[right] at (right laf8) {\Large \ce{^{1}_{}S_{0} La^{3+} }  } ;
\node[right] at (right cef8) {\Large \ce{^{2}_{}F_{5/2} Ce^{3+}  } } ;
\node[right] at (right prf8) {\Large \ce{^{3}_{}H_{4} Pr^{3+}  } } ;
\node[right] at (right ndf8) {\Large \ce{^{4}_{}I_{9/2} Nd^{3+}  } } ;

%\node[right] at (right 5d4) {\Large 5d} ;



\end{tikzpicture}

%\begin{tikzpicture}
%\calloutquote[author=Russelsunderskopplungsschema,width=0.5*\linewidth,position={(0,-1)},fill=green!30,rounded corners]{$\ce{^{2S+1}_{}L_J}$};
%\end{tikzpicture} 
\end{document}
