\documentclass{article}
\usepackage[german]{babel}

%\usepackage[activate={true,nocompatibility},final,tracking=true,kerning=true,spacing=true,factor=1100,stretch=10,shrink=10]{microtype}
%\usepackage{paratype}
\usepackage{libertine}
%\usepackage[scaled]{berasans}
\renewcommand*\familydefault{\sfdefault}  %%
\usepackage[utf8]{inputenc}

\usepackage[T1]{fontenc}

\usepackage{microtype}
\usepackage{xcolor}
\usepackage{pgffor}
\usepackage[version=4]{mhchem}
%\usetheme{beamertheme-bjeldbak/beamerthemebjeldbak}
\usepackage{tikz}
\usepackage{tikzorbital}
\usetikzlibrary{shapes.callouts} 

\pgfkeys{%
    /calloutquote/.cd,
    width/.code                   =  {\def\calloutquotewidth{#1}},
    position/.code                =  {\def\calloutquotepos{#1}}, 
    author/.code                  =  {\def\calloutquoteauthor{#1}},
    /calloutquote/.unknown/.code   =  {\let\searchname=\pgfkeyscurrentname
                                 \pgfkeysalso{\searchname/.try=#1,                                
    /tikz/\searchname/.retry=#1},\pgfkeysalso{\searchname/.try=#1,
                                  /pgf/\searchname/.retry=#1}}
                            }  


\newcommand\calloutquote[2][]{%
       \pgfkeys{/calloutquote/.cd,
         width               = 5cm,
         position            = {(0,-1)},
         author              = {}}
  \pgfqkeys{/calloutquote}{#1}                   
  \node [rectangle callout,callout relative pointer={\calloutquotepos},text width=\calloutquotewidth,/calloutquote/.cd,
     #1] (tmpcall) at (0,0) {#2};
  \node at (tmpcall.pointer){\calloutquoteauthor};    
}  


\begin{document}

\begin{tikzpicture}
%\draw [->,ultra thick] (-1,-2) --   (-1,4) node[above] {\Large Energie};
%
\drawLevel[pos = {(0,0)},    width = 1]{laf1};
\drawLevel[pos = {(1.2,0)},  width = 1]{laf2};
\drawLevel[pos = {(2.4,0)},  width = 1]{laf3};
\drawLevel[pos = {(3.6,0)},  width = 1]{laf4};
\drawLevel[pos = {(4.8,0)},  width = 1]{laf5};
\drawLevel[pos = {(6,0)},    width = 1]{laf6};
\drawLevel[pos = {(7.2,0)},  width = 1]{laf7};
%\calloutquote[author=Russelsunders,width=0.5*\linewidth,position={(0,-1)},fill=green!30,rounded corners]{$\ce{^{2S+1}_{}L_J}$};

\drawLevel[elec = up, pos = {(0,1)},    width = 1]{cef1};
\drawLevel[pos = {(1.2,1)},  width = 1]{cef2};
\drawLevel[pos = {(2.4,1)},  width = 1]{cef3};
\drawLevel[pos = {(3.6,1)},  width = 1]{cef4};
\drawLevel[pos = {(4.8,1)},  width = 1]{cef5};
\drawLevel[pos = {(6,1)},  width = 1]{cef6};
\drawLevel[pos = {(7.2,1)},  width = 1]{cef7};


\drawLevel[elec = up, pos = {(0,2)},    width = 1]{prf1};
\drawLevel[elec = up, pos = {(1.2,2)},  width = 1]{prf2};
\drawLevel[pos = {(2.4,2)},  width = 1]{prf3};
\drawLevel[pos = {(3.6,2)},  width = 1]{prf4};
\drawLevel[pos = {(4.8,2)},  width = 1]{prf5};
\drawLevel[pos = {(6,2)},  width = 1]{prf6};
\drawLevel[pos = {(7.2,2)},  width = 1]{prf7};


\drawLevel[elec = up, pos = {(0,3)},    width = 1]{ndf1};
\drawLevel[elec = up, pos = {(1.2,3)},  width = 1]{ndf2};
\drawLevel[elec = up, pos = {(2.4,3)},  width = 1]{ndf3};
\drawLevel[pos = {(3.6,3)},  width = 1]{ndf4};
\drawLevel[pos = {(4.8,3)},  width = 1]{ndf5};
\drawLevel[pos = {(6,3)},  width = 1]{ndf6};
\drawLevel[pos = {(7.2,3)},  width = 1]{ndf7};



\drawLevel[elec = up, pos = {(0,4)},    width = 1]{pmf1};
\drawLevel[elec = up, pos = {(1.2,4)},  width = 1]{pmf2};
\drawLevel[elec = up, pos = {(2.4,4)},  width = 1]{pmf3};
\drawLevel[elec = up,pos = {(3.6,4)},  width = 1]{pmf4};
\drawLevel[pos = {(4.8,4)},  width = 1]{pmf5};
\drawLevel[pos = {(6,4)},  width = 1]{pmf6};
\drawLevel[pos = {(7.2,4)},  width = 1]{pmf7};




\drawLevel[elec = up, pos = {(0,5)},    width = 1]{smf1};
\drawLevel[elec = up, pos = {(1.2,5)},  width = 1]{smf2};
\drawLevel[elec = up, pos = {(2.4,5)},  width = 1]{smf3};
\drawLevel[elec = up,pos = {(3.6,5)},  width = 1]{smf4};
\drawLevel[elec = up,pos = {(4.8,5)},  width = 1]{smf5};
\drawLevel[pos = {(6,5)},  width = 1]{smf6};
\drawLevel[pos = {(7.2,5)},  width = 1]{smf7};



\drawLevel[elec = up, pos = {(0,6)},    width = 1]{euf1};
\drawLevel[elec = up, pos = {(1.2,6)},  width = 1]{euf2};
\drawLevel[elec = up, pos = {(2.4,6)},  width = 1]{euf3};
\drawLevel[elec = up,pos = {(3.6,6)},  width = 1]{euf4};
\drawLevel[elec = up,pos = {(4.8,6)},  width = 1]{euf5};
\drawLevel[elec = up,pos = {(6,6)},  width = 1]{euf6};
\drawLevel[pos = {(7.2,6)},  width = 1]{euf7};



\drawLevel[elec = up, pos = {(0,7)},    width = 1]{gdf1};
\drawLevel[elec = up, pos = {(1.2,7)},  width = 1]{gdf2};
\drawLevel[elec = up, pos = {(2.4,7)},  width = 1]{gdf3};
\drawLevel[elec = up,pos = {(3.6,7)},  width = 1]{gdf4};
\drawLevel[elec = up,pos = {(4.8,7)},  width = 1]{gdf5};
\drawLevel[elec = up,pos = {(6,7)},  width = 1]{gdf6};
\drawLevel[elec = up,pos = {(7.2,7)},  width = 1]{gdf7};



%%%%%%







%\drawLevel[pos = {(0,2)},    width = 1]{5d0};
%\drawLevel[pos = {(1.2,2)},  width = 1]{5d1};
%\drawLevel[pos = {(2.4,2)},  width = 1]{5d2};
%\drawLevel[pos = {(3.6,2)},  width = 1]{5d3};
%\drawLevel[pos = {(4.8,2)},  width = 1]{5d4};
\node[right] at (right laf7) {\Large \ce{^{1}_{}S_{0}  }  } ;
\node[right] at (right cef7) {\Large \ce{^{2}_{}F_{5/2}  } } ;
\node[right] at (right prf7) {\Large \ce{^{3}_{}H_{4}   } } ;
\node[right] at (right ndf7) {\Large \ce{^{4}_{}I_{9/2}   } } ;
\node[right] at (right pmf7) {\Large \ce{^{5}_{}I_{4}   } } ;
\node[right] at (right smf7) {\Large \ce{^{6}_{}H_{5/2}   } } ;
\node[right] at (right euf7) {\Large \ce{^{7}_{}F_{0}   } } ;
\node[right] at (right gdf7) {\Large \ce{^{8}_{}S_{7/2}  } } ;

\node[left] at (left gdf1) {\Large \ce{ Gd^{3+}  } } ;
\node[left] at (left euf1) {\Large \ce{ Eu^{3+}  } } ;
\node[left] at (left smf1) {\Large \ce{ Sm^{3+}  } } ;
\node[left] at (left pmf1) {\Large \ce{ Pm^{3+}  } } ;
\node[left] at (left ndf1) {\Large \ce{ Nd^{3+}  } } ;
\node[left] at (left prf1) {\Large \ce{ Pr^{3+}  } } ;
\node[left] at (left cef1) {\Large \ce{ Ce^{3+}  } } ;
\node[left] at (left laf1) {\Large \ce{ La^{3+}  } } ;

\node[below,xshift=2mm] at (left laf1) {\Large \ce{ -3  } } ;

\node[below,xshift=2mm] at (left laf2) {\Large \ce{ -2  } } ;
\node[below,xshift=2mm] at (left laf3) {\Large \ce{ -1  } } ;
\node[below,xshift=4mm] at (left laf4) {\Large \ce{ 0  } } ;
\node[below,xshift=4mm] at (left laf5) {\Large \ce{ 1  } } ;
\node[below,xshift=4mm] at (left laf6) {\Large \ce{ 2  } } ;
\node[below,xshift=5mm] at (left laf7) {\Large \ce{ 3  } } ;


%\node[right] at (right 5d4) {\Large 5d} ;



\end{tikzpicture}

%\begin{tikzpicture}
%\calloutquote[author=Russelsunderskopplungsschema,width=0.5*\linewidth,position={(0,-1)},fill=green!30,rounded corners]{$\ce{^{2S+1}_{}L_J}$};
%\end{tikzpicture} 
\end{document}
