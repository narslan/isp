\documentclass{beamer}
\usepackage[german]{babel}
%\usepackage{libertine}
%\renewcommand*\familydefault{\sfdefault}  %%
\usepackage[default]{opensans}
%\usepackage{gillius2}
%\usepackage{lato}
\usepackage{siunitx}

\usepackage[utf8]{inputenc}

\usepackage[T1]{fontenc}

\usepackage{booktabs} % To thicken table lines
\usepackage{multirow}


\usepackage{microtype}
\usepackage{xcolor,multido,colortbl}
\usepackage{pgffor}

\usepackage{tikz}
\usepackage[version=4]{mhchem}
\usepackage{tikzorbital}

\usepackage{pgfplots}
\pgfplotsset{compat=1.13} 
 %\pgfplotsset{com  =1.13}
\usepackage{graphicx}
\usepackage{adjustbox}
\usepackage{hyperref}
\usepackage{caption}

\usepgfmodule{matrix} 
\usetikzlibrary{matrix,positioning}
\usetikzlibrary{arrows}
\usetikzlibrary{calc,shapes}
\usetikzlibrary{backgrounds}
\newcommand{\makemycolor}[2]{%
    \pgfmathsetmacro{\hue}{(#1/100)^1.715*0.79}%
    \definecolor{myhsbcolor}{hsb}{\hue,1,1}%
    \textcolor{myhsbcolor}{#2}%
}

\usepackage[absolute,overlay]{textpos}




\tikzset{
    level1/.style = {
        ultra thick,
        green
    },
    level4/.style = {
        ultra thick,
        blue
    },
    level5/.style = {
        ultra thick,
        red
    }
}

\newcommand\ytl[2]{
\parbox[b]{8em}{\hfill{\color{cyan}\bfseries\sffamily #1}~$\cdots\cdots$~}\makebox[0pt][c]{$\bullet$}\vrule\quad \parbox[c]{4.5cm}{\vspace{7pt}\color{red!40!black!80}\raggedright\sffamily #2.\\[7pt]}\\[-3pt]}
%\usepackage[texcoord,grid,gridunit=mm,gridcolor=red!10,subgridcolor=green!10]{eso-pic}
\setbeamertemplate{navigation symbols}{}


\title{Lumineszenz von Seltenerd-Ionen}
\author{Nevroz Arslan}

\begin{document}
  {%
    \setbeamertemplate{headline}{}
   \setbeamercolor{postit}{fg=black,bg=yellow}
    \frame{\titlepage}
  }

\begin{frame}[t]\frametitle{Was macht ein Leuchtstoff gut?}
Es kommt nicht darauf an!
\begin{itemize}
   \item hohe Quantenausbeute $QE = \frac{\text{emmitierte \ Photonen}}{\text{absorbierte Photonen}}\%$
   \item hohe Übergangswahrscheinlichkeit, kurze Lebensdauer und scharfe Emmision 
   \item geringe Neigung zu strahlungsloser Relaxation \ce{Ln^* -> Ln +} Wärme
\end{itemize}
   
\end{frame}

\begin{frame}[t]\frametitle{Was macht ein Leuchtstoff bunt?}
\begin{columns}
   \column{.7\textwidth}
    \begin{itemize}
    \item Farbe ist nicht gleich Wellenlänge
    \item Pink im sichtbaren Spektrum? 
    \item Es gibt andere Bestandteile
  % \item Lila ist dunkles Purpur.
    \end{itemize}
   \column{.3\textwidth}
      \begin{tikzpicture}
    \foreach \k in {0,1,...,100}{%
        \pgfmathsetmacro{\hue}{(\k/100)^1.715*0.79}
        \definecolor{mycolor}{rgb:hsb}{\hue,1,1}
        \node[color=mycolor] () at (0,\k/20) {$\bullet$};
    }%
    \foreach \f in {0,1,...,10}{%
        \pgfmathtruncatemacro{\num}{(\f*30)+480}
        \node () at (0.75,\f/2) {$\num$ nm};
    }
    \foreach \g/\h in {0/Rot,2/Orange,4/Gelb,6/Grün,8/Blau,10/Purpur}{%
        \pgfmathtruncatemacro{\num}{\g*10}
        \node at (-1,\g/2) {\makemycolor{\num}{\h}};
    }%
    \end{tikzpicture}

\end{columns}

\end{frame}


 
%Ein Farbraum beschreibt die Menge der darstellbaren Farben eines Mediums 

\begin{frame}[t]\frametitle{Woraus besteht Farbe?}
  Die drei natürlichen Koordinaten der Farbe
\begin{itemize}
      \item Farbton, (Wellenlänge,\textbf{hue}) %\multido{\nColr=0+1}{10}{\fcolorbox{white}{H!![\nColr]}{}}
      \begin{tikzpicture}
          \colorlet{color min hsb}[hsb]{red}
  \colorlet{color max hsb}[hsb]{magenta}
   \foreach \pos in {100,...,0}{
    \colorlet{my color hsb}[rgb]{color min hsb!\pos!color max hsb}
    \fill[fill=my color hsb,draw=none] (-3+\pos/20,0) rectangle +(5mm,3mm);
  }
      \end{tikzpicture}
   \item Sättigung (Weißanteil?) 
   \begin{itemize}
   \item Gesättigte Farben -> die spektrale Reinheit 
   \item Ungesättigte Farben sind unbunte Farben (schwarz, grau, weiß).
   \item Farben mit geringer Sättigung werden Pastellfarben genannt
   \end{itemize}
\begin{tikzpicture}
          \colorlet{color min hsb}[hsb]{white}
  \colorlet{color max hsb}[hsb]{red}
   \foreach \pos in {100,...,0}{
    \colorlet{my color hsb}[rgb]{color max hsb!\pos!}
    \fill[fill=my color hsb,draw=none] (-3+\pos/20,0) rectangle +(5mm,3mm);
  }
      \end{tikzpicture}


      \item Helligkeit (Intensität ) %\multido{\nColr=0+1}{10}{\fcolorbox{white}{B!![\nColr]}{}}
      \begin{itemize}
        \item  wird am model eines Schwarzkörpers analysert
      \end{itemize}
      
   \end{itemize}


 \end{frame}


\begin{frame}[t]\frametitle{Helligkeit am model eines Schwarzkörpers }
\begin{itemize}
  \item 
\end{itemize}

Die gesamte ausgestrahlte Energie $P$ [\si{\watt}] ...
     \begin{itemize}
     \item einer Oberflächen $A$
       \item proportional zur vierten Potenz der Temperatur
      \item  $\epsilon(T)$ ist materialabhängig
     \end{itemize}
   \begin{equation}
        P = \epsilon(T) \cdot \sigma \cdot A \cdot T^4
      \end{equation}

\begin{table}[!h]
\centering
      \begin{tabular}{ccc}
\toprule
\multirow{2}{*}{Material}&   Temperatur &  Emissivität  \\
& $T$ \si{\celsius}& $\epsilon(T)$\\
\midrule
 Wasser  & 10...50 &0,965  \\
 Sand & 20& 0,76\\
 Eisen,poliert & -73...727 & 0,32...0,60 \\
 Alufolie & 298 & 0.03 \\
\bottomrule
\end{tabular}
\caption{Beispiele für Emmisionsgrade einiger Oberflächen}
\end{table}
\end{frame}

\begin{frame}[t]\frametitle{Oberflächen aus Lanthanoiden}
    
\begin{figure}[!h]
\centering
      \includegraphics[scale=0.2]{pics/yagce.jpg}
      \caption*{\footnotesize \ce{(Y_{1-x}Gd_x)_3(Al_{1-y}Ga_y)_5O_{12}:Ce^{\cite{yagce}}} }
 \end{figure}
\end{frame}





%Mit zunehmender Temperatur verschiebt sich die maximale Strahlungsintensität eines Schwarzen Körpers zu kürzeren Wellenlängen, der Farbeindruck wechselt dabei vom Roten ins Bläulich-Weiße. Der Farbton einer (Wärme-) Lichtquelle lässt sich als Temperatur eines vergleichbaren Schwarzen Strahlers angeben. Damit erhält man die Farbtemperatur der Lichtquelle.
\begin{frame}[t]\frametitle{Farbtemperatur}
\begin{itemize}
  \item Plancksverteilungs
  \begin{itemize}
  \item \small Bei niedrigen Temperaturen wird hauptsächlich rotes Licht
  \item \small bei hohen Temperaturen blaues Licht abgestrahlt.
\end{itemize}

\end{itemize}


%\begin{figure}[!h]
%\centering

%\begin{tikzpicture}[scale=0.2]

% \begin{axis}[every axis plot post/.append style={
%   mark=none,samples=50,smooth}, % All plots: from -2:2, 50 samples, smooth, no marks
% axis x line*=bottom, % no box around the plot, only x and y axis
% axis y line*=left, % the * suppresses the arrow tips
% enlargelimits=upper] % extend the axes a bit to the right and top
% \addplot table[x=x,y=y] {pics/probe.dat};
% \end{axis}
% \end{tikzpicture}
%  \end{figure}


\begin{figure}[!h]
\centering
      \includegraphics[width=\textwidth]{pics/ct.png}
      \caption*{\footnotesize Farbtemperatur nach dem planckschen Strahlungsgesetz}
 \end{figure}


\end{frame}
   
\begin{frame}[t]\frametitle{Wiensches Verschiebungsgesetz: }
Die einfachste Verknüpfung der Strahlungsintensität mit der Wellenlänge     
\begin{equation}
    \lambda _{max}=\frac{2897,7 \mu m \cdot K}{T}
  \end{equation}
  \begin{itemize}
  \item \footnotesize $\lambda _{max}$ Wellenlänge $\mu m$, bei der die Intensität pro Wellenlängenintervall maximal ist
  \item \footnotesize  $T$: absolute Temperatur der strahlenden Fläche
\end{itemize}
\end{frame}



%Je höher die Temperatur eines Körpers ist, bei desto kürzeren Wellenlängen liegt das Maximum der Verteilung. 
%Die Wellenlänge maximaler Strahlungsleistung verschiebt sich also bei einer Temperaturänderung einfach umgekehrt proportional zur absoluten Temperatur des schwarzen Strahlers: Verdoppelt sich die Temperatur des Strahlers, so tritt die größte Strahlungsleistung bei der halben Wellenlänge auf.
\begin{frame}[t]\frametitle{Was macht ein Leuchtstoff umweltfreundlich?}
  \begin{beamerboxesrounded}[]{Umweltfreunlich}
  \begin{itemize}
   \item geringere UV-Anteile im Licht 
  \end{itemize}
  \end{beamerboxesrounded}
 
\end{frame}
 %Unter den Insekten sind vor allem Nachtfalter betroffen, deren Augen stark auf die UV-Strahlung von Hochdrucklampen reagieren. Die Gebäude- und Straßenlampen werden stundenlang umflogen, bis die Tiere ermatten oder vor Entkräftung sterben.https://de.wikipedia.org/wiki/Quecksilberdampflampe

\begin{frame}[t]\frametitle{Die Farben der Lanthanoiden}

\begin{beamerboxesrounded}[shadow=false]{}
  Die Lanthanoide treten bevorzugt dreiwertig auf.
\end{beamerboxesrounded}
     \begin{table}
      \centering
\resizebox{\linewidth}{!}{%
\begin{tabular}{cccccccccccccccc}
\toprule
 $[Xe]$ &\ce{La^{3+}} & \ce{Ce^{3+}} &\cellcolor[RGB]{ 110,194,47} \ce{Pr^{3+}} &\cellcolor[RGB]{202,99 ,138} \ce{Nd^{3+}} & \cellcolor[RGB]{ 240,42 ,242} \ce{Pm^{3+}}&\cellcolor[RGB]{  252, 255, 110} \ce{Sm^{3+}} & \ce{Eu^{3+}} & \ce{Gd^{3+}} & \ce{Tb^{3+}}& \cellcolor[RGB]{ 163, 255, 0} \ce{Dy^{3+}} &  \cellcolor[RGB]{  255,255, 0} \ce{Ho^{3+}} & \cellcolor[RGB]{  194,171, 187}\ce{Er^{3+}} & \cellcolor[RGB]{171,228,149} \ce{Tm^{3+}} & \ce{Yb^{3+}} & \ce{Lu^{3+}}\\
   & \cellcolor[RGB]{254,225,2} \ce{Ce^{4+}} & \cellcolor{yellow}\ce{Pr^{4+}} &\cellcolor[RGB]{ 111, 0,255} \ce{Nd^{4+}} &&  &  & \cellcolor[RGB]{151,26,14} \ce{Sm^{2+}} & \ce{Eu^{2+}} & \cellcolor[RGB]{ 251, 241, 35}\ce{Dy^{4+}}&  &  & &  &\cellcolor[RGB]{ 163,79,128}\ce{Tm^{2+}}&  \cellcolor[RGB]{181,201,19}\ce{Yb^{2+}}\\
   &  & &  &  &  &  &  &  & \cellcolor[RGB]{60,0,0} \ce{Tb^{4+}} &  &  &  &  & & \\
\midrule
  4f & 0 &1 & 2 & 3 & 4 & 5 & 6 & 7 & 8 & 9 & 10 & 11 & 12 & 13& 14\\
  \bottomrule
\end{tabular}
} 
\end{table}
 
    \begin{itemize}
      \item \ce{La} (\ce{f^0}), \ce{Gd} (\ce{f^7}), \ce{Lu} (\ce{f^{14}}) sind nur dreiwertig.
      \item Die stabilen f-Konfiguration werden erreicht in \ce{Ce^{4+}}, \ce{Tb^{4+}}, \ce{Eu^{2+}},\ce{Yb^{4+}}
    \end{itemize}
\end{frame}


\begin{frame}[t]\frametitle{Schalenaufbau}
    
   \begin{columns}
    \column{.8\textwidth}
      
    \begin{adjustbox}{max totalsize={.9\textwidth}{\textheight},center}
\begin{tikzpicture}
\draw [->,ultra thick] (0,0) --   (0,10) node[above] {\Large Energie};
\draw [->,ultra thick] (0,0) -- node[below] {\Large n (Schale)} (20,0);
    \draw[level4] (6,1) -- node[above] {4s} (7,1);
       \draw[level4] (6,2.5) -- node[above] {4p} (7,2.5);
       \draw[level4] (4.8,2.5) --  (5.8,2.5);
       \draw[level4] (7.2,2.5) --  (8.2,2.5);
      \draw[level4] (6,4) -- node[above] {4d} (7,4);
       \draw[level4] (7.2,4) --  (8.2,4);
       \draw[level4] (8.4,4) --  (9.4,4);
       \draw[level4] (4.8,4) --  (5.8,4);
       \draw[level4] (3.6,4) --  (4.6,4);
       \draw[level4] (6,6) -- node[above] {4f} (7,6);
       \draw[level4] (7.2,6) --  (8.2,6);
       \draw[level4] (8.4,6) --  (9.4,6);
       \draw[level4] (4.8,6) --  (5.8,6);
        \draw[level4] (3.6,6) --  (4.6,6);
        \draw[level4] (2.4,6) --  (3.4,6);
      \draw[level4] (9.6,6) --  (10.6,6);
      \draw[level5] (14,3)  -- node[above] {5s} (15,3);
       \draw[level5] (14,5) -- node[above] {5p} (15,5);
       \draw[level5] (12.8,5) --  (13.8,5);
       \draw[level5] (15.2,5) --  (16.2,5);
       \draw[level5] (14,6.5) -- node[above] {5d} (15,6.5);
       \draw[level5] (12.8,6.5) --  (13.8,6.5);
       \draw[level5] (15.2,6.5) --  (16.2,6.5);
        \draw[level5] (11.6,6.5) --  (12.6,6.5);
       \draw[level5] (16.4,6.5) --  (17.4,6.5);
       \draw[level1] (18,5.9) -- node[above] {6s} (19,5.9);
    \end{tikzpicture}
\end{adjustbox}
    \column{.2\textwidth}
     
\end{columns}

\end{frame}


\begin{frame}[t]\frametitle{RusselSanderskopplung}
\begin{itemize}
  \item Mehrelektronenzustände
  \item für einen gegebenen Wert von L existieren jeweils $2L+1$ entarteter Bandzustände 
  \item charakterisiert durch die magnetischen $M_L = 3,2,1,0,-1,-2,-3$ 
  \item S = Spinquantenzahl 
  \pause
  \item \begin{tikzpicture}[scale=0.8]

\drawLevel[elec = up, pos = {(0,3)},    width = 1]{ndf1};
\drawLevel[elec = up, pos = {(1.2,3)},  width = 1]{ndf2};
\drawLevel[elec = up, pos = {(2.4,3)},  width = 1]{ndf3};
\drawLevel[pos = {(3.6,3)},  width = 1]{ndf4};
\drawLevel[pos = {(4.8,3)},  width = 1]{ndf5};
\drawLevel[pos = {(6,3)},  width = 1]{ndf6};
\drawLevel[pos = {(7.2,3)},  width = 1]{ndf7};
\drawLevel[pos = {(8.4,3)},  width = 1]{ndf8};
\node[right] at (right ndf8) {\Large \ce{^{4}_{}I_{9/2} Nd^{3+}  } } ;

\end{tikzpicture}
\end{itemize}
%\ce{Y2O3:Eu^{2+}} bedeutet, dass ein Wirtgitter aus Yttriumoxid mit Europium-Ionen dotiert ist
% d. h. dass ca. fünf Prozent der Yttrium-Ionen durch Europium-Ionen ersetzt wurde.

\end{frame}

\begin{frame}[t]\frametitle{Interkombinationsverbot\cite{holeman}}
    %jeder Übergang, bei dem sich der sich Gesamtspin geändert, ist verboten 
\begin{figure}[!h]
\centering
\begin{tikzpicture}[scale=0.8]

\drawLevel[elec = updown, pos = {(0,3)},    width = 1]{ndf1};
\drawLevel[elec = up, pos = {(1.2,3)},  width = 1]{ndf2};
\drawLevel[elec = up, pos = {(2.4,3)},  width = 1]{ndf3};
\drawLevel[elec = up, pos = {(3.6,3)},  width = 1]{ndf4};
\drawLevel[elec = up, pos = {(4.8,3)},  width = 1]{ndf5};
\drawLevel[elec = up, pos = {(6,3)},  width = 1]{ndf6};
\drawLevel[elec = none, pos = {(7.2,3)},  width = 1]{ndf7};
\node[right] at (right ndf7) {\Large \ce{^{5}_{}S_{7/2} Gd^{3+}  } } ;


\drawLevel[elec = up, pos = {(0,1)},    width = 1]{ndf21};
\drawLevel[elec = up, pos = {(1.2,1)},  width = 1]{ndf2};
\drawLevel[elec = up, pos = {(2.4,1)},  width = 1]{ndf3};
\drawLevel[elec = up,pos = {(3.6,1)},  width = 1]{ndf4};
\drawLevel[elec = up,pos = {(4.8,1)},  width = 1]{ndf5};
\drawLevel[elec = up,pos = {(6,1)},  width = 1]{ndf6};
\drawLevel[elec = up,pos = {(7.2,1)},  width = 1]{ndf9};
\node[right] at (right ndf9) {\Large \ce{^{8}_{}S_{7/2} Gd^{3+}  } } ;
\node at (0.5,4) {$-3$};
\node at (1.5,4) {$-2$};
\node at (2.7,4) {$-1$};
\node at (4,4) {$0$};
\node at (5.5,4) {$1$};
\node at (6.5,4) {$2$};
\node at (7.5,4) {$3$};
\draw [<-, ultra thick] (left ndf1) edge[bend right]  (left ndf21);
\end{tikzpicture}
\caption{Interkombinationsverbot}

\end{figure}
\begin{itemize}
  \item Verboten, weil zu Spinpaarungen führt.
  \item die farblosen Beispiele aus ÜM: \ce{Fe^{3+}}, \ce{Mn^{2+}}
  \item farblos auch \ce{La^{3+} f^0}, \ce{Gd^{3+} f^7},\ce{Lu^{3+} f^{14}}   
\end{itemize}
\end{frame}
\begin{frame}[t]\frametitle{Regel von Laporte}
\begin{itemize}
  \item Übergänge zwischen unterschiedlicher Parität erlaubt
  \item s- und d-Orbitale sind gerade
  \item p- und d-Orbitale sind ungerade  
\end{itemize}
  
\end{frame}

\begin{frame}[t]\frametitle{\small Derzeit eingesetzte gute Leuchtstoffe }
  \begin{itemize}
    \item Das Drei-Banden-Konzept
          \begin{itemize}
            \item \footnotesize Fluoreszenzlampen mit sehr guter Farbwiedergabe und hoher Lichtausbeute
          \end{itemize}
  \end{itemize}

\begin{figure}[!h]
\centering
      \includegraphics[scale=0.2]{pics/dd.jpg}
      \caption*{\footnotesize \ce{Emission einer Drei-Banden-Leuchtstofflampe} }
 \end{figure}


\end{frame}

\begin{frame}[t]\frametitle{Eine Farbpalette}
    
    \begin{tabular}{lccccc}
\toprule
\multirow{2}{*}{Leuchtstoff}& Abk. &  \ce{$\lambda$} & Abs. bei  & QE & LE  \\
& &\si{\nano\meter}&254 \si{\nano\meter}  & \% & \si{\lumen\per\watt}\\
\midrule
 \ce{BaMgAl_{10}O_{17}:Eu}& BAM & \cellcolor[wave]{450} 450  & 90 &  90 &90\\
\ce{(Ce,Tb)MgAl_{11}O_{19}}   & CAT   &  \cellcolor[wave]{541} 541  &  95 & 90& 495\\
\ce{Y_2O_3:Eu}    & YOX   &  \cellcolor[wave]{611} 611  &  75 & 90 &280\\
\ce{LaPO_4:Ce,Tb}  & LAP  & \cellcolor[wave]{545} 545 & 95 &93 & 500\\
\ce{MgGePO_{5.5}F:Mn}  & MGM  & \cellcolor[wave]{655} 655 & 95 &80 & 80\\
\bottomrule
\end{tabular}

\begin{figure}[!h]
\centering
      \includegraphics[scale=0.2]{pics/palette.jpg}
      \caption*{\footnotesize \ce{Emission einer Drei-Banden-Leuchtstofflampe} }
 \end{figure}


\end{frame}


\begin{frame}[t]\frametitle{Konzentration Abhängigkeit }
\begin{itemize}
  \item Mit Leuchtstoffen gezielt verunreinigte Halbleiterkristalle
  \item It has been found that the average life period is practically independent of temperature and wavelength but depends markedly on concentration
  \item the intensity, polarisation and the average life period are all dependent on concentration
  \item 
\end{itemize}


\end{frame}

  \begin{frame}
  \frametitle{Quantenteiler}
 
  \end{frame}

\begin{frame}
  \frametitle{Stokes Verschiebung}
    \begin{itemize}
    \item 
    \end{itemize}
  \end{frame}


\begin{frame}[t]\frametitle{Was macht ein Leuchtstoff gut?}
\begin{itemize}
 \item \footnotesize Das von einer blauen oder UV-LED emittierte Licht ist
deutlich intensiver als dasjenige, das in einem Hg-Plasma
entsteht. 
 \item \footnotesize Quantenausbeute möglichst hoch sein soll
\end{itemize}
\end{frame}



\begin{frame}[t]\frametitle{Literaturverzeichnis}
    

\begin{thebibliography}{9}
\bibitem{yagce}
\emph{Yag-Ce}
 \url{http://www.scientificmaterials.com/products/ce-yag.php}.
\bibitem{wikilila}
\emph{Violett}
 \url{https://de.wikipedia.org/wiki/Violett}.
\bibitem{holeman}
E. Wiberg, \emph{Lehrbuch der Anorganische Chemie}, 102. Aufl., De Gruyter, Berlin \textbf{2007}, S. 1370.
\end{thebibliography}
  
\end{frame}


%http://www.rsc.org/chemistryworld/News/2009/September/01090901.asp
%https://www.osapublishing.org/oe/fulltext.cfm?uri=oe-19-S3-A331
\end{document}

% Reference