\documentclass{beamer}
\usepackage[german]{babel}
%\usepackage{libertine}
%\renewcommand*\familydefault{\sfdefault}  %%
\usepackage[default]{opensans}
%\usepackage{gillius2}
%\usepackage{lato}
\usepackage{siunitx}

\usepackage[utf8]{inputenc}

\usepackage[T1]{fontenc}

\usepackage{booktabs} % To thicken table lines
\usepackage{multirow}


\usepackage{microtype}
\usepackage{xcolor,multido,colortbl}
\usepackage{pgffor}

\usepackage{tikz}
\usepackage[version=4]{mhchem}
\usepackage{tikzorbital}

%\usepackage{pgfplots}
 %\pgfplotsset{compat=1.13}
\usepackage{graphicx}
\usepackage{adjustbox}
\usepackage{hyperref}
\newcommand{\makemycolor}[2]{%
    \pgfmathsetmacro{\hue}{(#1/100)^1.715*0.79}%
    \definecolor{myhsbcolor}{hsb}{\hue,1,1}%
    \textcolor{myhsbcolor}{#2}%
}

\usepackage[absolute,overlay]{textpos}
%\usecolortheme[light,accent=red]{solarized}
%\definecolorseries{H}{hsb}{last}[hsb]{0,1,1}[hsb]{1,1,1}\resetcolorseries[10]{H}
%\definecolorseries{S}{hsb}{last}[hsb]{0.1,0,0}[hsb]{.1,1,1}\resetcolorseries[10]{S}
%\definecolorseries{B}{hsb}{last}[hsb]{1,1,0}[hsb]{1,1,1}\resetcolorseries[10]{B}
\tikzset{
    level1/.style = {
        ultra thick,
        green
    },
    level4/.style = {
        ultra thick,
        blue
    },
    level5/.style = {
        ultra thick,
        red
    }
}

\newcommand\ytl[2]{
\parbox[b]{8em}{\hfill{\color{cyan}\bfseries\sffamily #1}~$\cdots\cdots$~}\makebox[0pt][c]{$\bullet$}\vrule\quad \parbox[c]{4.5cm}{\vspace{7pt}\color{red!40!black!80}\raggedright\sffamily #2.\\[7pt]}\\[-3pt]}
%\usepackage[texcoord,grid,gridunit=mm,gridcolor=red!10,subgridcolor=green!10]{eso-pic}
\setbeamertemplate{navigation symbols}{}


\title{Lumineszenz von Seltenerd-Ionen}
\author{Nevroz Arslan}

\begin{document}
  {%
    \setbeamertemplate{headline}{}
   \setbeamercolor{postit}{fg=black,bg=yellow}
    \frame{\titlepage}
  }

\begin{frame}[t]\frametitle{Was macht ein Leuchtstoff gut?}
Es kommt nicht darauf an!
\begin{itemize}
   \item hohe Quantenausbeute $QE = \frac{\text{emmitierte \ Photonen}}{\text{absorbierte Photonen}}\%$
   \item hohe Übergangswahrscheinlichkeit, kurze Lebensdauer und scharfe Emmision 
   \item geringe Neigung zu strahlungsloser Relaxation \ce{Ln^* -> Ln +} Wärme
\end{itemize}
   
\end{frame}

\begin{frame}[t]\frametitle{Was macht ein Leuchtstoff gut?}
\begin{columns}
   \column{.7\textwidth}
    \begin{itemize}
    \item Es kommt darauf an! 
   \item Farbe ist nicht gleich Wellenlänge
   \item Pink im sichtbaren Spektrum? 
   \item Lila ist dunkles Purpur.
    \end{itemize}
   \column{.3\textwidth}
      \begin{tikzpicture}
    \foreach \k in {0,1,...,100}{%
        \pgfmathsetmacro{\hue}{(\k/100)^1.715*0.79}
        \definecolor{mycolor}{rgb:hsb}{\hue,1,1}
        \node[color=mycolor] () at (0,\k/20) {$\bullet$};
    }%
    \foreach \f in {0,1,...,10}{%
        \pgfmathtruncatemacro{\num}{(\f*30)+480}
        \node () at (0.75,\f/2) {$\num$ nm};
    }
    \foreach \g/\h in {0/Rot,2/Orange,4/Gelb,6/Grün,8/Blau,10/Purpur}{%
        \pgfmathtruncatemacro{\num}{\g*10}
        \node at (-1,\g/2) {\makemycolor{\num}{\h}};
    }%
    \end{tikzpicture}

\end{columns}

\end{frame}

\begin{frame}[t]\frametitle{}
    
\begin{beamerboxesrounded}[shadow=true]{Was macht ein Leuchtstoff gut?}
      Ein quantitatives Maß ist der Farbwiedergabeindex 

     \pgfdeclareimage[interpolate=true,height=6cm]{image1}{pics/cie}
     \pgfuseimage{image1}

   \end{beamerboxesrounded}

\end{frame}
   

 
%Ein Farbraum beschreibt die Menge der darstellbaren Farben eines Mediums 

\begin{frame}[t]\frametitle{Die drei natürlichen Koordinaten der Farbe}
    

\begin{itemize}
      \item Farbton (Wellenlänge) %\multido{\nColr=0+1}{10}{\fcolorbox{white}{H!![\nColr]}{}}
      \item Sättigung (Weißanteil) %\multido{\nColr=0+1}{10}{\fcolorbox{white}{S!![\nColr]}{}}
      \item Helligkeit (Intensität ) %\multido{\nColr=0+1}{10}{\fcolorbox{white}{B!![\nColr]}{}}
    \end{itemize}
 \end{frame}


\begin{frame}[t]\frametitle{Was macht ein Leuchtstoff gut}
  \begin{beamerboxesrounded}[]{Umweltfreunlich}
  \begin{itemize}
   \item geringere UV-Anteile im Licht 
   \item nicht toxisch sein 
  \end{itemize}
  \end{beamerboxesrounded}
 
\end{frame}
 %Unter den Insekten sind vor allem Nachtfalter betroffen, deren Augen stark auf die UV-Strahlung von Hochdrucklampen reagieren. Die Gebäude- und Straßenlampen werden stundenlang umflogen, bis die Tiere ermatten oder vor Entkräftung sterben.https://de.wikipedia.org/wiki/Quecksilberdampflampe

\begin{frame}[t]\frametitle{Die Farben der Lanthanoiden}

\begin{beamerboxesrounded}[shadow=false]{}
  Die Lanthanoide treten bevorzugt dreiwertig auf.
\end{beamerboxesrounded}
     \begin{table}
      \centering
\resizebox{\linewidth}{!}{%
\begin{tabular}{cccccccccccccccc}
\toprule
 $[Xe]$ &\ce{La^{3+}} & \ce{Ce^{3+}} &\cellcolor[RGB]{ 110,194,47} \ce{Pr^{3+}} &\cellcolor[RGB]{202,99 ,138} \ce{Nd^{3+}} & \cellcolor[RGB]{ 240,42 ,242} \ce{Pm^{3+}}&\cellcolor[RGB]{  252, 255, 110} \ce{Sm^{3+}} & \ce{Eu^{3+}} & \ce{Gd^{3+}} & \cellcolor[RGB]{171,228,149} \ce{Tb^{3+}}& \ce{Dy^{3+}} & \ce{Ho^{3+}} & \cellcolor[RGB]{  194,171, 187}\ce{Er^{3+}} & \ce{Tm^{3+}} & \ce{Yb^{3+}} & \ce{Lu^{3+}}\\
   & \cellcolor[RGB]{254,225,2} \ce{Ce^{4+}} & \cellcolor{yellow}\ce{Pr^{4+}} & \ce{Nd^{4+}} &\ce{La^{3+}} & \ce{La^{3+}} & \ce{La^{3+}} & \cellcolor[RGB]{151,26,14} \ce{Sm^{2+}} & \ce{Eu^{2+}} & \ce{Dy^{4+}}&  &  & &  & \ce{Tm^{2+}}& \ce{Yb^{2+}}\\
   &  & &  &  &  &  &  &  & \ce{Tb^{4+}} &  &  &  &  & & \\
\midrule
  4f & 0 &1 & 2 & 3 & 4 & 5 & 6 & 7 & 8 & 9 & 10 & 11 & 12 & 13& 14\\
  \bottomrule
\end{tabular}
} 
\end{table}
 
    \begin{itemize}
      \item \ce{La} (\ce{f^0}), \ce{Gd} (\ce{f^7}), \ce{Lu} (\ce{f^{14}}) sind nur dreiwertig.
      \item Die stabilen f-Konfiguration werden erreicht in \ce{Ce^{4+}}, \ce{Tb^{4+}}, \ce{Eu^{2+}},\ce{Yb^{4+}}
    \end{itemize}
\end{frame}


\begin{frame}[t]\frametitle{Schalenaufbau}
    
   \begin{columns}
    \column{.5\textwidth}
      
    \begin{adjustbox}{max totalsize={.9\textwidth}{\textheight},center}
\begin{tikzpicture}
\draw [->,ultra thick] (0,0) --   (0,10) node[above] {\Large Energie};
\draw [->,ultra thick] (0,0) -- node[below] {\Large n (Schale)} (20,0);
    \draw[level4] (6,1) -- node[above] {4s} (7,1);
       \draw[level4] (6,2.5) -- node[above] {4p} (7,2.5);
       \draw[level4] (4.8,2.5) --  (5.8,2.5);
       \draw[level4] (7.2,2.5) --  (8.2,2.5);
      \draw[level4] (6,4) -- node[above] {4d} (7,4);
       \draw[level4] (7.2,4) --  (8.2,4);
       \draw[level4] (8.4,4) --  (9.4,4);
       \draw[level4] (4.8,4) --  (5.8,4);
       \draw[level4] (3.6,4) --  (4.6,4);
       \draw[level4] (6,6) -- node[above] {4f} (7,6);
       \draw[level4] (7.2,6) --  (8.2,6);
       \draw[level4] (8.4,6) --  (9.4,6);
       \draw[level4] (4.8,6) --  (5.8,6);
        \draw[level4] (3.6,6) --  (4.6,6);
        \draw[level4] (2.4,6) --  (3.4,6);
      \draw[level4] (9.6,6) --  (10.6,6);
      \draw[level5] (14,3)  -- node[above] {5s} (15,3);
       \draw[level5] (14,5) -- node[above] {5p} (15,5);
       \draw[level5] (12.8,5) --  (13.8,5);
       \draw[level5] (15.2,5) --  (16.2,5);
       \draw[level5] (14,6.5) -- node[above] {5d} (15,6.5);
       \draw[level5] (12.8,6.5) --  (13.8,6.5);
       \draw[level5] (15.2,6.5) --  (16.2,6.5);
        \draw[level5] (11.6,6.5) --  (12.6,6.5);
       \draw[level5] (16.4,6.5) --  (17.4,6.5);
       \draw[level1] (18,5.9) -- node[above] {6s} (19,5.9);
    \end{tikzpicture}
\end{adjustbox}
    \column{.5\textwidth}
    \begin{itemize}
    %alle Elemente enthaltenzwei s-Elektronen in der äußersten (6.) Schale
    % jeweils zwei s-Elektronen und sechs p-Elektronen in der zweitäußersten (5.) Schale
  \item \ce{6s^2} 
  \item \ce{5s^2} und \ce{5p^6} 
\end{itemize}    
\end{columns}

\end{frame}







\begin{frame}[t]\frametitle{Dotierung}
\ce{Y2O3:Eu^{2+}} bedeutet, dass ein Wirtgitter aus Yttriumoxid mit Europium-Ionen dotiert ist
 d. h. dass ca. fünf Prozent der Yttrium-Ionen durch Europium-Ionen ersetzt wurde.


\end{frame}


\begin{frame}[t]\frametitle{Dotierung und Wirtgitter}
\begin{itemize}
  \item Mit Leuchtstoffen gezielt verunreinigte Halbleiterkristalle
  \item It has been found that the average life period is practically independent of temperature and wavelength but depends markedly on concentration
  \item the intensity, polarisation and the average life period are all dependent on concentration
\end{itemize}


\end{frame}

  \begin{frame}
  \frametitle{Lumineszenz}
    \begin{itemize}
    \item Der eigentliche Lumineszenzprozeß ist weitgehend unabhängig von der chemischen Umgebung
    \item Strahlungslose Desaktivierung \ce{Ln^* -> Ln +} Wärme
    \item Lumineszenz \ce{Ln^* -> Ln + h \nu}
    \item VUV Photonen (145-180 nm)
    \end{itemize}
  \end{frame}

\begin{frame}
  \frametitle{Stokes Verschiebung}
    \begin{itemize}
    \item Ausgestrahltes Licht besitzt eine kleinere Frequenz als absorbierte Strahlung
    \end{itemize}
  \end{frame}


\begin{frame}[t]\frametitle{Was macht ein Leuchtstoff gut?}
\begin{itemize}
 \item Anregungsenergie möglichst effizient absorbieren
 \item Quantenausbeute möglichst hoch sein soll
\end{itemize}
\end{frame}

  %Folie 4 Auswahlregeln
  \begin{frame}[t]\frametitle{Laporte Verbot}

\end{frame}
\section{}
  \begin{frame}[t]\frametitle{Farben von Lanthanoid-Ionen in wässriger Lösung}
\begin{itemize}
  \item In Verbindungen liegt Lanthan als farbloses \ce{La^{3+}} vor.
\end{itemize}

\end{frame}


\begin{frame}[t]\frametitle{Literaturverzeichnis}
    

\begin{thebibliography}{9}

\bibitem{wikilila}
\emph{Violett}
 \url{https://de.wikipedia.org/wiki/Violett}.

\end{thebibliography}
  


\end{frame}


%http://www.rsc.org/chemistryworld/News/2009/September/01090901.asp
%https://www.osapublishing.org/oe/fulltext.cfm?uri=oe-19-S3-A331
\end{document}

% Reference