\documentclass{beamer}
\usepackage[german]{babel}
%\usepackage{libertine}
%\renewcommand*\familydefault{\sfdefault}  %%
\usepackage[default]{opensans}
%\usepackage{gillius2}
%\usepackage{lato}

\usepackage[utf8]{inputenc}

\usepackage[T1]{fontenc}

\usepackage{booktabs} % To thicken table lines
\usepackage{multirow}


\usepackage{microtype}
\usepackage{xcolor}
\usepackage{pgffor}

\usepackage{tikz}
\usepackage[version=4]{mhchem}

%\usepackage{pgfplots}
 %\pgfplotsset{compat=1.13}
\usepackage{graphicx}

\usepackage[absolute,overlay]{textpos}
%\usecolortheme[light,accent=red]{solarized}

\newcommand\ytl[2]{
\parbox[b]{8em}{\hfill{\color{cyan}\bfseries\sffamily #1}~$\cdots\cdots$~}\makebox[0pt][c]{$\bullet$}\vrule\quad \parbox[c]{4.5cm}{\vspace{7pt}\color{red!40!black!80}\raggedright\sffamily #2.\\[7pt]}\\[-3pt]}
%\usepackage[texcoord,grid,gridunit=mm,gridcolor=red!10,subgridcolor=green!10]{eso-pic}
\setbeamertemplate{navigation symbols}{}


\title{Lumineszenz von Seltenerd-Ionen}
\author{Nevroz Arslan}

\begin{document}
  {%
    \setbeamertemplate{headline}{}
   \setbeamercolor{postit}{fg=black,bg=yellow}
    \frame{\titlepage}
  }
\section{Einsatzbereiche seltener Erden als Leuchtstoffe} % (fold)

% section section_name (end)
  \begin{frame}[t]\frametitle{Einsatzbereiche seltener Erden als Leuchtstoffe}

  \begin{beamerboxesrounded}[shadow=true]{}
    Einzigartige optischen Eigenschaften
  \end{beamerboxesrounded}
  \begin{beamerboxesrounded}[shadow=false]{}
    Vielfache Einsatzmöglichkeiten als lumineszierende Materialien.
  \end{beamerboxesrounded}
\pause
    \begin{itemize}
      \item Leuchtstoffröhren
      \item Farbfernsehröhren
      \item Anzeigetafeln
      \item Plasmabildschirmen
      \item Die weißen Leuchtdiode (LED)
      \item Na- und Hg-Dampflampen
      \item Geldscheine
      \item Kunstgläser, Pigment
      \item Upconversion-Leuchtstoffen
      \item Strahlentherapie zur Behandlung von Hautkrankheiten
    \end{itemize}
  \end{frame}

\section{Gruppeneigenschaften} % (fold)

\begin{frame}[t]\frametitle{Elektronenkonfiguration}

\begin{beamerboxesrounded}[shadow=true]{Dreiwertigkeit}
  Die Lanthanoide treten bevorzugt dreiwertig auf.
\end{beamerboxesrounded}
     \begin{table}
      \centering
\resizebox{\linewidth}{!}{%
\begin{tabular}{cccccccccccccccc}
\toprule
 $[Xe]$ &\ce{La^{3+}} & \ce{Ce^{3+}} & \ce{Pr^{3+}} & \ce{Nd^{3+}} & \ce{Pm^{3+}}& \ce{Sm^{3+}} & \ce{Eu^{3+}} & \ce{Gd^{3+}} & \ce{Tb^{3+}}& \ce{Dy^{3+}} & \ce{Ho^{3+}} & \ce{Er^{3+}} & \ce{Tm^{3+}} & \ce{Yb^{3+}} & \ce{Lu^{3+}}\\
   & \ce{Ce^{4+}} &\ce{Pr^{4+}} & \ce{Nd^{4+}} &\ce{La^{3+}} & \ce{La^{3+}} & \ce{La^{3+}} & \ce{Sm^{2+}} & \ce{Eu^{2+}} & \ce{Dy^{4+}}&  &  & &  & \ce{Tm^{2+}}& \ce{Yb^{2+}}\\
   &  & &  &  &  &  &  &  & \ce{Tb^{4+}} &  &  &  &  & & \\
\midrule
  4f & 0 &1 & 2 & 3 & 4 & 5 & 6 & 7 & 8 & 9 & 10 & 11 & 12 & 13& 14\\
  \bottomrule
\end{tabular}
}
\end{table}

    \begin{itemize}
      \item \ce{La} (\ce{f^0}), \ce{Gd} (\ce{f^7}), \ce{Lu} (\ce{f^{14}}) sind nur dreiwertig.
      \item Die stabilen f-Konfiguration werden erreicht in \ce{Ce^{4+}}, \ce{Tb^{4+}}, \ce{Eu^{2+}},\ce{Yb^{4+}}
    \end{itemize}
\end{frame}



  \begin{frame}[t]\frametitle{ Die farblosen Lanthanoidionen }
  \begin{beamerboxesrounded}[shadow=false]{\ce{Gd^{3+}, 4f^7, ^{8}_{}S_{7/2}}}
  \begin{itemize}
  \item \ce{Gd^{3+}} hat keine Farbe
  \end{itemize}
    \end{beamerboxesrounded}
  \end{frame}

 \begin{frame}[t]\frametitle{ Die farbigen Lanthanoidionen }
  \begin{beamerboxesrounded}[shadow=false]{\ce{Tb^{3+}, 4f^8, ^{7}_{}F_{6} -> ^{5}_{}D_{4}}}
   \pgfdeclareimage[interpolate=true,height=1cm]{image1}{pics/tbsulfat}
   \pgfuseimage{image1}
  \end{beamerboxesrounded}
  \end{frame}


\begin{frame}[t]\frametitle{Dotierung}
\ce{Y2O3:Eu^{2+}} bedeutet, dass ein Wirtgitter aus Yttriumoxid mit Europium-Ionen dotiert ist
 d. h. dass ca. fünf Prozent der Yttrium-Ionen durch Europium-Ionen ersetzt wurde.


\end{frame}


\begin{frame}[t]\frametitle{Dotierung und Wirtgitter}
\begin{itemize}
  \item Mit Leuchtstoffen gezielt verunreinigte Halbleiterkristalle
  \item It has been found that the average life period is practically independent of temperature and wavelength but depends markedly on concentration
  \item the intensity, polarisation and the average life period are all dependent on concentration
\end{itemize}


\end{frame}

  \begin{frame}
  \frametitle{Lumineszenz}
    \begin{itemize}
    \item Der eigentliche Lumineszenzprozeß ist weitgehend unabhängig von der chemischen Umgebung
    \item Strahlungslose Desaktivierung \ce{Ln^* -> Ln +} Wärme
    \item Lumineszenz \ce{Ln^* -> Ln + h \nu}
    \end{itemize}
  \end{frame}

\begin{frame}
  \frametitle{Stokes Verschiebung}
    \begin{itemize}
    \item Ausgestrahltes Licht besitzt eine kleinere Frequenz als absorbierte Strahlung
    \end{itemize}
  \end{frame}


\begin{frame}[t]\frametitle{Was macht ein Leuchtstoff gut?}
\begin{itemize}
 \item Anregungsenergie möglichst effizient absorbieren
 \item Quantenausbeute möglichst hoch sein soll
\end{itemize}
\end{frame}

  %Folie 4 Auswahlregeln
  \begin{frame}[t]\frametitle{Laporte Verbot}

\end{frame}
\section{}
  \begin{frame}[t]\frametitle{Farben von Lanthanoid-Ionen in wässriger Lösung}
\begin{itemize}
  \item In Verbindungen liegt Lanthan als farbloses \ce{La^{3+}} vor.
\end{itemize}

\end{frame}
%http://www.rsc.org/chemistryworld/News/2009/September/01090901.asp
%https://www.osapublishing.org/oe/fulltext.cfm?uri=oe-19-S3-A331
\end{document}

