\documentclass[12pt]{article}
\usepackage{amsmath,mathtools}
\usepackage[usenames,dvipsnames]{xcolor}
\usepackage[bitstream-charter]{mathdesign}
\usepackage{microtype}
\usepackage[utf8]{inputenc}
\usepackage[T1]{fontenc}
\usepackage{libertine}
%\usepackage[libertine]{newtxmath}
%\usepackage{graphicx}
\usepackage{siunitx}
\usepackage[german]{babel}
%\usepackage{tikz}
\usepackage{chemfig}
\usepackage{fancyhdr}
\usepackage{sectsty}
\usepackage{setspace}
\usepackage[compatibility=4.7,language=german]{chemmacros}
\pagestyle{fancy}
\lhead{Nevroz Arslan }
\rhead{20.11.2015}
\setlength{\headheight}{15pt}


%\renewcommand*\printatom[1]{{\fontsize{10}{12}\selectfont\ensuremath{\mathsf{#1}}}}
%\sectionfont{\fontsize{12}{15}\selectfont}
\setcrambond{2pt}{}{}

%\renewcommand*\printatom[1]{{\fontsize{10}{12}\selectfont\ensuremath{\mathsf{#1}}}}
%\sectionfont{\fontsize{12}{15}\selectfont}
\setdoublesep{0.35700 em}  % 'Bond Spacing'
\setatomsep{1.78500 em}    % 'Fixed Length'
\setbondoffset{0.18265 em} % 'Margin Width'
\newcommand{\bondwidth}{0.06642 em} % 'Line Width'
\setbondstyle{line width = \bondwidth}

\begin{document}

%\chemrel[\chemfig{H_2O}][\chemfig{Hg{(OAc)}_2}]{->}
\section{Präparat 4: \textnormal{Tetrakis(trimethylsilyl)silane}}
%\schemedebug{true}
\schemestart
4x \chemfig{Si(-[3,0.7]{H_3C})(<:[:200]{H_3C})(<[:250]{H_3C})-[,0.6]Cl}
\+
\chemfig{8 \ Li}
\+
\chemfig{SiCl_4}
\arrow(.mid east--[yshift=-10pt]){->[THF]}[,1.5]
\chemfig{Si(-[3]{(H_3C)}_3Si)(<:[:200]{(H_3C)}_3Si)(<[:250]{(H_3C)}_3Si)-[:-1]{Si{(CH_3)}_3}}
\+
\chemfig{8 \ LiCl}
\schemestop
\section{Berechnung des Ansatzes: }
\normalsize \section{Durchführung \cite{organikum}}
\section{Ausbeute}
\section{Physikalische Daten des Produktes}
\section{Spektrenauswertung}
\section{Mechanismus\cite{bio}}
\setatomsep{2.5em}
%\schemedebug{true}
\bigskip

\schemestart
\chemname{\chemfig{@{sib0}Si(-[3]{Cl})(<:[:200]{Cl})(<[:250]{Cl})-[@{clb}]@{cl}Cl}}{I}
\+
\chemfig{@{lir0}\lewis{4.,Li}} x 2
\arrow(c1.mid east--c2.mid west)
\chemname{\chemfig{\chemabove{\lewis{0:,Si}}{\hspace{5mm}\scriptstyle\ominus}(-[3]{Cl})(<:[:200]{Cl})(<[:250]{Cl})-[7,0.6,,,draw=none]\chemabove{Li}{\hspace{5mm}\scriptstyle\oplus}}}{II}
\+
\chemfig{LiCl}
\chemmove{\draw[shorten <=3pt,shorten >=1pt,red,thick](clb).. controls +(north:5mm) and +(90:5mm).. (cl);}
\chemmove{\draw[shorten <=5pt,shorten >=1pt,red,thick](lir0).. controls +(220:10mm) and +(300:5mm)..node[below,blue,yshift=-1mm] {$\scriptstyle 2 e^-$} (sib0);}
\schemestop
\bigskip

\schemestart
\chemname{\chemfig{@{sib1}\chemabove{\lewis{0:,Si}}{\hspace{5mm}\scriptstyle\ominus}(-[3]{Cl})(<:[:200]{Cl})(<[:250]{Cl})-[7,0.6,,,draw=none]\chemabove{Li}{\hspace{5mm}\scriptstyle\oplus}}}{II}
\+
\chemname{\chemfig{@{sib2}Si(-[3]{H_3C})(<:[:200]{H_3C})(<[:250]{H_3C})-[@{clb1}]@{cl1}Cl}}{III}
\arrow(c1.mid east--c2.mid west)
\chemname{\chemfig{Si(-[3]{Cl})(<:[:200]{Cl})(<[:250]{Cl})-Si(-[1]{CH_3})(<:[:-10]{CH_3})(<[:-70]{CH_3})}}{IV}
\+
\chemfig{LiCl}
\chemmove{\draw[shorten <=6pt,shorten >=2pt,red,thick](sib1).. controls +(30:9mm) and +(170:5mm).. (sib2);}
\chemmove{\draw[shorten <=3pt,shorten >=1pt,red,thick](clb1).. controls +(100:5mm) and +(90:5mm).. (cl1);}
\schemestop
\bigskip

\schemestart
\chemname{\chemfig{Si(-[3]{H_3C})(<:[:200]{H_3C})(<[:250]{H_3C})-@{sia3}Si(-[@{clb3}1]@{cla3}Cl)(<:[:-10]Cl)(<[:-70]Cl)}}{IV}
\+
\chemfig{@{lir1}\lewis{4.,Li}} x 2
\arrow(c1.mid east--c2.mid west)
\chemname{\chemfig{Si(-[3]{H_3C})(<:[:200]{H_3C})(<[:250]{H_3C})-\chemabove{\lewis{2:,Si}}{\hspace{5mm}\scriptstyle\ominus}(-[1,0.8,,,draw=none]\chemabove{Li}{\hspace{5mm}\scriptstyle\oplus})(<:[:-10]Cl)(<[:-70]Cl)}}{V}
\+
\chemfig{LiCl}
\chemmove{\draw[shorten <=3pt,shorten >=1pt,red,thick](clb3).. controls +(north west:5mm) and +(145:5mm).. (cla3);}
\chemmove{\draw[shorten <=5pt,shorten >=5pt,red,thick](lir1).. controls +(230:10mm) and +(330:5mm)..node[below,blue,yshift=-1mm] {$\scriptstyle 2 e^-$} (sia3);}
\schemestop

\bigskip

\schemestart
\chemname{\chemfig{Si(-[3]{H_3C})(<:[:200]{H_3C})(<[:250]{H_3C})-@{sib5}\chemabove{\lewis{2:,Si}}{\hspace{5mm}\scriptstyle\ominus}(-[2,0.8,,,draw=none]\chemabove{Li}{\hspace{5mm}\scriptstyle\oplus})(<:[:-10]Cl)(<[:-70]Cl)}}{V}
\+
\chemname{\chemfig{@{sib6}Si(-[3]{H_3C})(<:[:200]{H_3C})(<[:250]{H_3C})-[@{clb6}]@{cl6}Cl}}{III}
\arrow(c1.mid east--c2.mid west)
\chemname{\chemfig{Si(-[3]{H_3C})(<:[:200]{H_3C})(<[:250]{H_3C})-Si(-[1]Si|{(CH_3)}_3)(<:[:-10]Cl)(<[:-70]Cl)}}{VI}
\+
\chemfig{LiCl}
\chemmove{\draw[shorten <=6pt,shorten >=2pt,red,thick](sib5).. controls +(north:9mm) and +(170:5mm).. (sib6);}
\chemmove{\draw[shorten <=3pt,shorten >=1pt,red,thick](clb6).. controls +(100:5mm) and +(90:5mm).. (cl6);}
\schemestop

\bigskip

\schemestart
\chemname{\chemfig{Si(-[3]{H_3C})(<:[:200]{H_3C})(<[:250]{H_3C})-@{sia4}Si(-[1]Si|{(CH_3)}_3)(<:[@{clb7}:-10]@{cla7}Cl)(<[:-70]Cl)}}{VI}
\+
\chemfig{@{lir2}\lewis{4.,Li}} x 2
\arrow(c1.mid east--c2.mid west)
\chemname{\chemfig{Si(-[3]{H_3C})(<:[:200]{H_3C})(<[:250]{H_3C})-\chemabove{\lewis{2:,Si}}{\hspace{5mm}\scriptstyle\ominus}(-[1,0.8,,,draw=none]\chemabove{Li}{\hspace{5mm}\scriptstyle\oplus})(<:[:-10]Si|{(CH_3)}_3)(<[:-70]Cl)}}{VII}
\+
\chemfig{LiCl}
\chemmove{\draw[shorten <=3pt,shorten >=1pt,red,thick](clb7).. controls +(north west:5mm) and +(145:5mm).. (cla7);}
\chemmove{\draw[shorten <=5pt,shorten >=5pt,red,thick](lir2).. controls +(230:10mm) and +(330:5mm)..node[below,blue,yshift=-1mm] {$\scriptstyle 2 e^-$} (sia4);}
\schemestop

\bigskip

\schemestart
\chemname{\chemfig{Si(-[3]{H_3C})(<:[:200]{H_3C})(<[:250]{H_3C})-@{sil7}\chemabove{\lewis{2:,Si}}{\hspace{5mm}\scriptstyle\ominus}(-[2,0.8,,,draw=none]\chemabove{Li}{\hspace{5mm}\scriptstyle\oplus})(<:[:-10]Si|{(CH_3)}_3)(<[:-70]Cl)}}{VII}
\+
\chemname{\chemfig{@{sib7}Si(-[3]{H_3C})(<:[:200]{H_3C})(<[:250]{H_3C})-[@{clb8}]@{cl8}Cl}}{III}
\arrow(c1.mid east--c2.mid west)
\chemname{\chemfig{Si(-[:120]{(CH_3)}_3|Si)(<:[:200]{(CH_3)}_3|Si)(<[:250]{(CH_3)}_3|Si)-Cl}}{VIII}
\+
\chemfig{LiCl}
\chemmove{\draw[shorten <=6pt,shorten >=2pt,red,thick](sil7).. controls +(north:9mm) and +(170:5mm).. (sib7);}
\chemmove{\draw[shorten <=3pt,shorten >=1pt,red,thick](clb8).. controls +(100:5mm) and +(90:5mm).. (cl8);}
\schemestop

\bigskip

\schemestart
\chemname{\chemfig{@{sia5}Si(-[:120]{(CH_3)}_3|Si)(<:[:200]{(CH_3)}_3|Si)(<[:250]{(CH_3)}_3|Si)-[@{clc7}]@{cld7}Cl}}{VIII}
\+
\chemfig{@{lir3}\lewis{4.,Li}} x 2
\arrow(c1.mid east--c2.mid west)
\chemname{\chemfig{\chemabove{\lewis{0:,Si}}{\hspace{5mm}\scriptstyle\ominus}(-[:120]{(CH_3)}_3|Si)(<:[:200]{(CH_3)}_3|Si)(<[:250]{(CH_3)}_3|Si)(-[1,0.8,,,draw=none]\chemabove{Li}{\hspace{5mm}\scriptstyle\oplus})}}{IX}
\+
\chemfig{LiCl}
\chemmove{\draw[shorten <=3pt,shorten >=1pt,red,thick](clc7).. controls +(north west:5mm) and +(145:5mm).. (cld7);}
\chemmove{\draw[shorten <=5pt,shorten >=5pt,red,thick](lir3).. controls +(230:10mm) and +(330:5mm)..node[below,blue,yshift=-1mm] {$\scriptstyle 2 e^-$} (sia5);}
\schemestop

\bigskip

\schemestart
\chemname{\chemfig{@{sil9}\chemabove{\lewis{0:,Si}}{\hspace{5mm}\scriptstyle\ominus}(-[:120]{(CH_3)}_3|Si)(<:[:200]{(CH_3)}_3|Si)(<[:250]{(CH_3)}_3|Si)(-[1,0.8,,,draw=none]\chemabove{Li}{\hspace{5mm}\scriptstyle\oplus})}}{IX}
\+
\chemname{\chemfig{@{sib9}Si(-[3]{H_3C})(<:[:200]{H_3C})(<[:250]{H_3C})-[@{clb9}]@{cl9}Cl}}{III}
\arrow(c1.mid east--c2.mid west)
\chemname{\chemfig{Si(-[:120]{(CH_3)}_3|Si)(<:[:200]{(CH_3)}_3|Si)(<[:250]{(CH_3)}_3|Si)(-[,1.3]Si{(CH_3)}_3)}}{X}
\+
\chemfig{LiCl}
\chemmove{\draw[shorten <=6pt,shorten >=2pt,red,thick](sil9).. controls +(20:9mm) and +(170:5mm).. (sib9);}
\chemmove{\draw[shorten <=3pt,shorten >=1pt,red,thick](clb8).. controls +(100:5mm) and +(90:5mm).. (cl8);}
\schemestop


\section{Literatur}
\renewcommand{\section}[2]{}%
\begin{thebibliography}{}
\bibitem{organikum}
H. Becker \textit{Organikum}, 21. Aufl., Wiley-VCH, Weinheim \textbf{2009}, S. 528.
\bibitem{bio}
J. Buddrus, \textit{Grundlagen der Organische Chemie}, 4. Aufl., De Gruyter, Berlin \textbf{2011}, S. 635.
\end{thebibliography}
\end{document}